\documentclass[letterpaper]{article}
\usepackage[utf8]{inputenc}
\usepackage[brazil]{babel}
\usepackage[pdftex]{graphicx}
\usepackage[top=1.70in,left=1.70in,right=1.75in,bottom=1.75in]{geometry}
\usepackage[usenames,dvipsnames,dvips]{color}
\usepackage{verbatim}
\usepackage{mdwlist}
\usepackage{listings}
\renewcommand{\baselinestretch}{1.3}

\begin{document}

\section*{Proposal For}

\subsection*{"The Lettuce Book: acceptance testing in Python"}
\normalsize\textbf{by Gabriel Falcão} \large{/} \texttt{gabriel@lettuce.it}

\normalsize

This book is intended to be a full guide to lettuce, suitable for
either developers or product managers.

\noindent
Developers will be blind tasters of all lettuce internals, and product
managers will learn how to write down their specifications through
storytelling.

\noindent
From the development perspective this will cover: hooks, project-wide
helpers, and a full guide to browser testing.

\noindent
This book revolves around four main subjects:

\begin{itemize}

\item{Coding, coding, coding: The reader will be always practicing with examples}

\item{Scaling a codebase: using tests to leverage the organization of the code}

\item{Taking the maximum advantage of Django + Lettuce + Browser Testing}

\item{Rethinking Pythonism: How automated testing can influentiate in conventions, and how to work around it}
\end{itemize}

\noindent

The book starts by getting absolute beginners excited about testing, showing how easy is to write the first test.
It then goes further with more complex examples, always emphasizing
tips to write more descriptive tests, introduce storytelling.

\noindent
At that point product managers will be able to understand how to write
stories as lettuce scenarios, and help developers to focus on their
goals for the product.

\noindent
The next step is getting into web development with Django, showing how
lettuce integrates perfectly with it, and how to take maximum
advantage of that, going further on outside-in web development with
Lettuce, Django\footnote{one of the most known web frameworks for python: http://djangoproject.com} and Splinter\footnote{A browser-driver that works with Firefox and Chrome: http://splinter.cobrateam.info/}.

\section*{Chapters Thought}
This is the very first proposal chapter. It mostly covers my
thoughts for what a full guide on BDD in Python should contain.

\subsection*{Introduction to Lettuce}
\noindent
Introduces what is Lettuce, why it was made, who should use it, and
whether or not it's a good choice for testing your application.

\subsection*{Getting Started}
\noindent
This chapter starts with a simple guide on how to prepare the environment, using straightforward tool such as PIP\footnote{A hassless python package manager}.

Here we will be driving the reader to the computer to try out what he
learned in the very first page.

\subsection*{Motivation for testing}
\noindent
Explains the benefits of testing, rambling on how Lettuce is not a
silver bullet, and how it is supposed to work together with other
testing tools.
\noindent
This is one of the most important parts, I've came along with
consultancy for some companies and I've seen how lettuce has been
misused, this chapter will clarify that.

\subsection*{Rethinking the Pythonism}
\noindent
In this chapter I propose the reader to rethink python community's
notorious buzzword: pythonism

\noindent
Python wealths a PEP\footnote{Acronym for Python Enhancement Proposals} named \textit{"The Zen of Python"}\footnote{http://www.python.org/dev/peps/pep-0020/}, dated of 2004 with some statements that have been taken almost as the "b10 commandments".

\noindent
Well the times changed since then, and this chapter will show how
automated testing influence on that, always pointing out Lettuce by
example.

\subsection*{Outside-in development with Django, Lettuce and Splinter}
\noindent
This chapter shows outside-in development in practice, here is where
developers and product managers will start talking smoothly.
Developers will learn how to build up a new web project from the
ground, shaping specifications into tests and making them pass.

\noindent
Here is where developers will see themselves shipping pieces of work in a timely fashion.

\subsection*{A World of hooks: Anatomy of Terrain and World}
\noindent
This chapter will cover all the internals of Lettuce: Hooks, World,
Terrain.  After going through out this chapter, the reader will become
a Lettuce expert, how to decouple reusable test helpers into a
project-wide place, in a very organized way.

\subsection*{Debugging}
\noindent
When we speak about Test-driven and Behaviour-driven development,
tests are not supposed to pass in the first round, but to get passing
in the last cycle.
\noindent
But the problem is when they break, some wrong code was commited to
the version control repository, how do I debug it?
\noindent
This chapter will show all the caveats on debugging lettuce tests.
\subsection*{Continuous Integration}
\noindent
This chapter shows how important is to have a server running the tests
continuously, how to set up lettuce builds in any CI server, showing
examples for Build Bot, Jenkins, and Cruise Control.

\section*{Competing Books }
\noindent
This proposed book will cover Behaviour-driven development, continuous
integration and storytelling with Python, I could find no titles in
the market that cover this field, and absolutely none covering
Lettuce as a whole.

\noindent
Although there is a title "Python Testing Cookbook" work in progress,
and the author, Greg Turnquist, blogged\footnote{http://pythontestingcookbook.posterous.com/good-by-pycukes-hello-lettuce} saying that he is going to cover
"Acceptance Testing" with Lettuce.

\section*{Marketing Information}

Lettuce, the tool that outlines the subjects covered by the book, is
being used by more and more companies either big ones like the
brazillian Globo.com, Mozilla and a crescent number of startups like
Yipit and Votizen.

\noindent
"Googling" after "BDD in Python" you will see blog entries and
presentations about usages of Lettuce.

\noindent
In September 2010 I gave a lightning talk in DjangoCon2010 in order to introduce Lettuce to the Django community.
In September 2011, Adam Nelson, CTO of the startup Yipit, gave a techtalk in DjangoCon2011 named "Testing with Lettuce and Splinter"\footnote{http://djangocon.us/schedule/presentations/67/}
\section*{Promotional Ideas}

I will put ads on Lettuce's oficial documentation website:
http://lettuce.it, as well as on my blog.

\noindent
Thus, I will end all of my tech talks mentioning the book as a full
guide to Lettuce for Acceptance testing and BDD.
\end{document}
